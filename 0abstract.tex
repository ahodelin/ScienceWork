\begin{abstract}
\noindent\textbf{Introduction}: The increase of size and complexity of simulation studies in systems biology and systems medicine proposes new challenges to sharing reproducible results. The \ac{combine} archive improves the coordination of standard formats for several features of simulation studies \cite{combine}. On the other side, GitHub has been used as an essential common platform for managing software projects and supporting collaborative development \cite{github}. In this case study we aimed to develop a fully-featured  \ac{combine} archive of a dynamic pathway model of JAK/STAT5 signaling \cite{bachmannmodel} using agile co-working technologies.\\
\textbf{Methods}: In this case study we implemented a GitHub environment to create a publicly traceable progress documentation of a \ac{combine} archive assembly. Systematic searches for available modeling files (\ac{sbml}) and software tools were conducted. The interaction between the tools and the modeling data was assessed and a fitting combination was chosen. Additionally we created the graphical notation files (\ac{sbgnml} and \ac{sbgn}), that are needed to visualize the simulation results. In a next step, we reproduced a substantial part of the original simulation data using both established software tools (COPASI) but also the python-based tool Tellurium \cite{tellurium}. Finally, a \ac{combine} Archive was created using the CAT tool and was made publicly available.\\
\textbf{Results}: We chose a GitHub repository as collaborative working environment with content versioning, code deposition and development as well as a wiki documentation. This delivered a prompt and sound scaffold for the development of the \ac{combine} archive. By systematic searches we identified the \textsc{bachmann2011.xml} (BIOMD0000000861) as the appropriate model file and were able to add the needed \ac{sbgnml} and \ac{sbgn} code. To fuse the above simulation components, a Tellurium \cite{tellurium} script was developed to reproduce the numeric and graphical results of the original paper. Finally, we created and validated a \ac{combine} archive comprehensively assembling all information necessary to reproduce the majority of figures from Bachmann \textit{et al.}\cite{bachmannmodel}.\\
\textbf{Conclusions}: In this case study we demonstrated the value of agile co-working for the development of a fully-featured combine archive of the Bachmann model with a low-threshold approach. Although we could not reproduce the full extend of the simulations given in the original work due to missing details in the original publication, we were able to create a fully functional archive that improves the reproducibility and accessibility of the original research results. \\
\end{abstract}