\section*{Results} \label{sec:resuslt}
\subsection*{\acl{sbml} results}
In our systematic search we found five available SBML-Bachmann models, one of them as support information of Bachmann \textit{et. al.} \cite{bachmannmodel}, this was the first delivered model. The others originated  from two different repositories, \hyperlink{https://www.systemsmedicine.net/posts/jws-online-biological-systems-modelling}{JWS Online} and \hyperlink{https://www.ebi.ac.uk/biomodels/}{BioModels}.

We found three models on BioModels. The first model, \hyperlink{https://www.ebi.ac.uk/biomodels/model/download/BIOMD0000000347.2?filename=BIOMD0000000347_url.xml}{\textsf{BIOMD0000000347\_url.xml}}, was submitted on 22nd July 2011 and modified on 31st January 2012. Together with this model were other files in different formats. Most of them were generated by tools to simulate, visualize, validate and document the model, one of them is another SBML model (\hyperlink{https://www.ebi.ac.uk/biomodels/model/download/BIOMD0000000347.2?filename=BIOMD0000000347_url.xml}{\textsf{BIOMD0000000347\_urn.xml}}). 

The third and newest, \hyperlink{https://www.ebi.ac.uk/biomodels/model/download/BIOMD0000000861.2?filename=Bachmann2011.xml}{\textsf{Bachmann2011.xml}}, was posted on 14th November 2019. This file contained other complementary files for the simulation of this model. The models provided in JWS Online do not have any date of building or update, so it was not possible to know when these were built. 

The first model in JWS online, \hyperlink{https://jjj.bio.vu.nl/models/bachmann/sbml/?download=1}{\textsf{bachmann.xml}}, is from \textit{Mus musculus} and represents the STAT's pathway in a cell simulation \textit{in silico}. 

The second model, \hyperlink{https://jjj.bio.vu.nl/models/bachmann2/sbml/?download=1}{\textsf{bachmann2.xml}}, was obtained from the BioModels database (BioModels ID: BIOMD0000000347).

\subsection*{\acl{sbgn} results}

\subsection*{\acl{sedml} results}

\subsection*{Unexpected events and observations}
\begin{itemize}
    \item Some required parameters to simulate specific experimental conditions were not included in the model.
\item Some model outputs are defined by observation functions and were not included as parameters in the model itself. It probably would be possible to define these parameters either in the \ac{sbml} file or in the \ac{sedml} files of the experiment based on information included in the supplementary material. However, we did not address this issue given the limited time frame of this project.
\item Several experiments presented in the paper are based on testing a range of Epo levels. We could not fully reproduce these experiments, because it was not clear how to simulate different Epo concentrations using the model.
\item The knockout of CIS and/or SOCS3 could not be simulated using the information provided in the paper.
\end{itemize}

\subsection*{Reproducibility}
At the end of our project, we could build a fully-featured \ac{combine} archive and integrate the most generated files, reports and pictures of the Bachmann model in the web platform CombineArchiveWeb.