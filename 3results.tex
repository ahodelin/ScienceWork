\section*{Results} \label{sec:resuslt}
\subsection*{\acf{sbml} results}
In our systematic search we found five available SBML-Bachmann models, one of them as support information of Bachmann \textit{et al.} \cite{bachmannmodel}, this was the first delivered model. The others originated  from two different repositories, \hyperlink{https://www.systemsmedicine.net/posts/jws-online-biological-systems-modelling}{JWS Online} and \hyperlink{https://www.ebi.ac.uk/biomodels/}{BioModels}.

We found three models on BioModels. The first model, \hyperlink{https://www.ebi.ac.uk/biomodels/model/download/BIOMD0000000347.2?filename=BIOMD0000000347_url.xml}{'BIOMD0000000347\_url.xml'}, was submitted on 22nd July 2011 and modified on 31st January 2012. Together with this model were other files in different formats. Most of them were generated by tools to simulate, visualize, validate and document the model, one of them is another SBML model (\hyperlink{https://www.ebi.ac.uk/biomodels/model/download/BIOMD0000000347.2?filename=BIOMD0000000347_url.xml}{'BIOMD0000000347\_urn.xml'}). 

The third and newest, \hyperlink{https://www.ebi.ac.uk/biomodels/model/download/BIOMD0000000861.2?filename=Bachmann2011.xml}{'Bachmann2011.xml'}, was posted on 14th November 2019. This file contained other complementary files for the simulation of this model. The models provided in JWS Online do not have any date of building or update, so it was not possible to know when these were built. 

The first model in JWS online, \hyperlink{https://jjj.bio.vu.nl/models/bachmann/sbml/?download=1}{'bachmann.xml'}, is from \textit{Mus musculus} and represents the STAT's pathway in a cell simulation \textit{in silico}. 

The second model, \hyperlink{https://jjj.bio.vu.nl/models/bachmann2/sbml/?download=1}{'bachmann2.xml'}, was obtained from the BioModels database (BioModels ID: BIOMD0000000347).

\subsection*{\acf{sbgn} results}
Figurative representation of data is a key factor to the rapid perception of information from it. The provision of the graphical representation of a model in a standardized, unambiguous form fosters the reusability and exchangeability of the model. The publication of the Bachmann model contains a process diagram of the model, which not uses a standard graphical notation.

Since no off-the-shelf standardised graphical representation could be found we decided to create our own SBGN map from scratch. Based on the authors´ diagram choice in the publication and the suggestions of Le Novère \textit{et al.} \cite{sbgnnovere} and Touré \textit{et al.} \cite{sbgntoure} we chose the 'Process Description' language for our COMBINE archive. Following the selection of the \ac{sbgn} language, we identified several useful tools and ultimately created the \ac{pd} map with 'Newt Editor'. 

To confirm the validity of the developed SBGN we used the integrated LibSBGN-based validation feature of 'Newt Editor' as well as imported the SBGN-ML into VANTED/SBGN-ED, Krayon for SBGN and SBGNViz. \textbf{Figure 1} shows the visually improved map after cleaning it up and colouring the most important features of the model.

\subsection*{\acf{sedml} results}

\subsubsection*{Selection of a SED-ML Tool}

Our search revealed five software tools for the simulation of biological systems:
\begin{enumerate} 
    \item \hyperlink{http://copasi.org/}{\textbf{\ac{copasi}}}\\A software application for the simulation and analysis of biochemical networks and their dynamics \cite{copasi}.
   
    \item \hyperlink{https://www.systemsmedicine.net/posts/jws-online-biological-systems-modelling}{\textbf{JWS Online}}\\ Systems Biology tool for the construction, modification, and simulation of kinetic models and the storage of curated models \cite{jwsonline}. On this repository was one of the found models.
   
    \item \hyperlink{https://sed-ml.org/}{\textbf{\ac{swt}}}\\A suite of tools for creating, editing, simulating and validating \ac{sedml} files \cite{sedml}. 
   
    \item \hyperlink{https://tellurium.readthedocs.io/en/latest/}{\textbf{Tellurium}}\\ A tool to model, simulate and analyze biochemical systems \cite{tellurium}.
    
\end{enumerate}

\subsubsection*{Generation of SED-ML files}

Following a review of available tools for creating SED-ML files \cite{sedmltool} we decided on two separate approaches. Firstly, we conducted default simulation in SED-ML WebTools as a basis for experiment-specific simulations in COPASI (as described by Scharm \textit{et al.} \cite{combine}). Secondly, we performed experiment-specific simulations in Tellurium, a Python platform for systems biology \cite{tellurium}. Tellurium includes a number of pre-installed Python libraries, plugins and tools for biological modelling and can be used via any Python frontend. We used Google Colaboratory (Colab), a cloud-based Jupyter notebook service for collaborative Python programming. The detailled documentation to the scripts are provided on our GitHub-Repository.

\subsubsection*{Simulation results}
Figure shows the 

The created SED-ML files were validated in SED-ML WebTools and integrated into the COMBINE archive. Subsequently the output files were created by simulating all SED-ML files within the COMBINE archive using Tellurium and loading individual SED-ML files in COPASI or SED-ML WebTools. 


\subsection*{Unexpected events and observations}
Interestingly some required parameters to simulate specific experimental conditions were not included in the model. That way some model outputs are defined by observation functions that were not included as parameters in the model itself. 

The plots for pJAK2, pEPOR and tSTAT5 (e.g., Fig3, Fig4) could not be generated since these parameters are missing from the model itself. Information on how to derive these observables from existing model parameters is included in the supplementary material (e.g., Supplementary Information, page 21). 

In contrast to CIS and SOCS3, the overexpression of SHP1 could not be reproduced since the corresponding parameter (SHP1oe) was not included in the model (nor in any other Bachmann model files available online). We did not modify the model in order to simulate different level of Epo as described in the paper. While the corresponding figure legends and supplementary material contain some information on Epo concentrations used for these experiments, it was not clear ho to simulate this in the model.

Many figures (e.g. Figure 4) contain experimental data in addition to model simulations. Although this data is referenced as supplementary material in the online version of the paper, we did not include this in our plots. The ranges of the y axes do not correspond to the figures in the paper. 
Several experiments presented in the paper are based on testing a range of \ac{epo} levels. We were not able to reproduce these experiments, because it was not clear how to simulate different \ac{epo} concentrations using the model.

Finally, we uploaded all data from our GitHub repository into the platform \textit{CombineArchiveWeb}, except generated XML files report during the validations, because, the \textit{CombineArchiveWeb} interface launches the error: \textsf{Unknown Error: Cannot upload file}, therefore we created a new ticket-issue in the GitHub repository from the \textit{CombineArchiveWeb} project.

\subsection*{Reproducibility}
As mentioned in the results section there have been some flaws in the reproducibility due to missing data given in the original work \cite{bachmannmodel}. Nevertheless at the end of our project, we could build a fully-featured \ac{combine} archive and integrate the generated files, reports and pictures of the Bachmann model in the web platform CombineArchiveWeb.