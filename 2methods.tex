\section*{Materials and methods} \label{matmet}

In this case study we aimed to generate a fully featured \ac{combine} Archive of a dynamic pathway model investigating the role of suppressor of cytokine (SOCS) family members in JAK2/STAT5 signaling \cite{bachmannmodel}. Generation of the \ac{combine} Archive was performed in accordance with a recently published guideline from Schwarm and Waltemath \cite{combine}.

The purpose of this project was divided into four groups. The tasks of each of them were the management of a documentation platform, review of exiting materials, comparison of provided models, graphical representation, and supply of a model script.

\subsection*{Set up of a collaborative working environment}
To provide the model from Bachmann \textit{et. al.}\cite{bachmannmodel} as a Fully Featured \ac{combine} Archive, we have created a public repository using the open-source platform GitHub, with a CC0-1.0 License. We chose GitHub as a data management platform to supervise the course of the project because it provides an intuitive and easy customizable environment, along with some features for documentation, and agile project management \cite{github}. This repository contains the proposal directory's structure from Scharm \& Waltemath \cite{combine} with the directories documentation, model, experiment, and result.

To achieve this goal, we research the literature about the Bachmann model and the \ac{combine} Archive, along with modeling file formats, checking the reproducibility of the \ac{combine} structure and software tools that had been used. In addition, we established communication channels, developed a rough schedule, provided a template for documentation, and periodically reviewed intermediate deliverables.

\subsection*{Model selection}
In our research, we found six \ac{sbml}-Bachmann models in two different repositories, \hyperlink{https://www.systemsmedicine.net/posts/jws-online-biological-systems-modelling}{JWS Online} and \hyperlink{https://www.ebi.ac.uk/biomodels/}{BioModels} (see Results). We chose the last model from 14th November 2019 to perform other tasks in our project, because it provides complementary files for the simulation.

\subsection*{Software tools and Versions}
We found and tested five software tools for the simulation of biological systems:
\begin{enumerate} 
    \item \hyperlink{https://www.systemsmedicine.net/posts/jws-online-biological-systems-modelling}{JWS Online}, Systems Biology tool for the construction, modification, and simulation of kinetic models and the storage of curated models \cite{jwsonline}. On this repository was one of the found models.
    \item \hyperlink{https://cat.bio.informatik.uni-rostock.de/}{Webviewer Uni Rostock} (CombineArchiveWeb), a tool to visualize and manage \ac{combine} files \cite{combine}; this is the goal application of our project. 
    \item \hyperlink{http://copasi.org/}{\ac{copasi}}, a software application for the simulation and analysis of biochemical networks and their dynamics \cite{copasi}.
    \item \hyperlink{https://sed-ml.org/}{\ac{sedml}}, a suite of tools for creating, editing, simulating and validating \ac{sedml} files \cite{sedml}. 
    \item \hyperlink{https://tellurium.readthedocs.io/en/latest/}{Tellurium},  a tool to model, simulate and analyze biochemical systems \cite{tellurium}.
\end{enumerate}

\subsection*{Graphical representation}
One of the objectives of our project was to provide a standardized graphical representation of the Bachmann model based on the \ac{sbgn}. We performed research but could not find any \ac{sbgn} of this model. Therefore, we decided to create an \ac{sbgn} network \textit{de novo} based on Le Novère \cite{sbgnnovere} and Touré \textit{et. al.} \cite{sbgntoure}. In this step, we selected the \ac{sbgn} language, and lastly, we created the \ac{pd} map with the web tool Newt Editor (v3.0.3) \cite{newteditor}. To validate the \ac{sbgnml} we, imported it into several software and libraries including LibSBGN from Newt Editor, the open-source software \ac{vanted} \cite{vanted}, Krayon for \ac{sbgn}\cite{krayon}, and SBGNViz \cite{sbgnviz}. Lastly, we cleaned up the map and colored the relevant features in the model to improve the developed map.

\subsection*{\acs{sedml} Generation and Validation}
For the generation of new \ac{sedml} files to reproduce selected experiments performed by Bachmann \textit{et. al.}, we made use of two open source software tools:
\begin{enumerate}
    \item default simulation in \hyperlink{http://sysbioapps.spdns.org/SED-ML_Web_Tools}{\ac{sedml} WebTools} as a basis for experiment-specific simulations in \hyperlink{http://copasi.org/}{\ac{copasi}} (as described by Scharm \& Waltemath \cite{combine})
    \item experiment-specific simulations in Tellurium, a Python platform for systems biology \cite{tellurium}
\end{enumerate}

The created \ac{sedml} files were validated in \ac{sedml} WebTools and integrated into the \ac{combine} archive. Subsequently, output files were created by

\begin{enumerate}
    \item simulating all \ac{sedml} files within the \ac{combine} archive using Tellurium and
    \item loading individual \ac{sedml} files in \ac{copasi} or \ac{sedml} WebTools, and included in the \ac{combine} archive.
\end{enumerate}

\subsection*{Creation of the \acs{combine} Archive}
