\section*{Materials and methods} \label{sec:matmet}

The development of the \ac{combine} Archive was performed in accordance with a recently published guideline from Schwarm and Waltemath \cite{combine}.

The tasks of this project were distributed on  four teams and split into the sections 
\begin{itemize}
    \item Setup of an agile working environment
    \item Systematic review of existing materials and comparison of provided models
    \item Graphical representation
    \item Supply of a model script
\end{itemize}

\subsection*{Setup of an agile working environment} \label{setupagwork}
To provide the model from Bachmann \textit{et. al.} \cite{bachmannmodel} as a fully featured \ac{combine} archive, we created a public repository using the open-source platform GitHub, with a \ac{cc}0-1.0-license. We chose GitHub as a data management platform to supervise the course of the project as it provides an intuitive and easily customizable environment, along with features for documentation, and agile project management \cite{github}. This repository contains the proposal directory's structure from Scharm \& Waltemath \cite{combine} with the following directories:
\begin{itemize}
    \item Documentation (files describing the model and its characteristics)
    \item Model (files describing and encoding the biological system, e.g. \ac{sbml} format)
    \item Experiment (files describing and encoding the experimental setup, e.g. \ac{sedml})
    \item Result (files obtained from running \textit{in silico} experiments, including plots and tables)
\end{itemize}

%\textit{DIESER ABSCHNITT GEHÖRT HIER IRGENDWIE NICHT HIN ?!\\
%To achieve this goal, we research the literature about the Bachmann model and the %\ac{combine} Archive, along with modeling file formats, checking the %reproducibility of the \ac{combine} structure and software tools that had been %used. In addition, we established communication channels, developed a rough %schedule, provided a template for documentation, and periodically reviewed %intermediate deliverables.}

\subsection*{Systematic review of existing materials and comparison of provided models}
In our systematic search, we found five \ac{sbml}-Bachmann models in two different repositories, \hyperlink{https://www.systemsmedicine.net/posts/jws-online-biological-systems-modelling}{JWS Online} and \hyperlink{https://www.ebi.ac.uk/biomodels/}{BioModels} (see Baseline data). We chose the latest model dating from 14th November 2019 as it provided complementary files for the simulation.

\subsubsection*{Software tools and Versions}
Our search revealed five software tools for the simulation of biological systems:
\begin{enumerate} 
    \item \hyperlink{http://copasi.org/}{\textbf{\ac{copasi}}}\\A software application for the simulation and analysis of biochemical networks and their dynamics \cite{copasi}.
   
    \item \hyperlink{https://www.systemsmedicine.net/posts/jws-online-biological-systems-modelling}{\textbf{JWS Online}}\\ Systems Biology tool for the construction, modification, and simulation of kinetic models and the storage of curated models \cite{jwsonline}. On this repository was one of the found models.
   
    \item \hyperlink{https://sed-ml.org/}{\textbf{\ac{swt}}}\\A suite of tools for creating, editing, simulating and validating \ac{sedml} files \cite{sedml}. 
   
    \item \hyperlink{https://tellurium.readthedocs.io/en/latest/}{\textbf{Tellurium}}\\ A tool to model, simulate and analyze biochemical systems \cite{tellurium}.
    
     \item \hyperlink{https://cat.bio.informatik.uni-rostock.de/}{\textbf{Webviewer Uni Rostock}} (CombineArchiveWeb)\\ A tool to visualize and manage \ac{combine} files \cite{combine}; this is the target application of our project. 
\end{enumerate}

\subsection*{Graphical representation}
One of the objectives of our project was to provide a standardized graphical representation of the Bachmann model based on the \ac{sbgn}. We performed research but could not find any \ac{sbgn} of this model. Therefore, we decided to create an \ac{sbgn} network \textit{de novo} based on previous works of Le Novère \cite{sbgnnovere} and Touré \textit{et. al.} \cite{sbgntoure}. In this step, we selected the \ac{sbgn} language, and lastly created the \ac{pd} map with the web tool Newt Editor (v3.0.3) \cite{newteditor}. 

To validate the \ac{sbgnml} we, imported it into several software and libraries including LibSBGN from Newt Editor, the open-source software \ac{vanted} \cite{vanted}, Krayon for \ac{sbgn} \cite{krayon}, and SBGNViz \cite{sbgnviz}. Lastly, we cleaned up the map and colored the relevant features in the model to improve the developed map.

\subsection*{Supply of a model script}
\textsf{this section needs maintainance. soll das wirklich hier so stehen? oder ist das der abschnitt über die auswahl des richtigen files?}
\subsubsection*{\acs{sedml} Generation and Validation}
For the development of new \ac{sedml} files in order to reproduce the simulations performed by Bachmann \textit{et. al.}, we made use of two open source software tools:
\begin{enumerate}
    \item default simulation in \hyperlink{http://sysbioapps.spdns.org/SED-ML_Web_Tools}{\ac{sedml} WebTools} as a basis for experiment-specific simulations in \hyperlink{http://copasi.org/}{\ac{copasi}} (as described by Scharm \& Waltemath \cite{combine})
    \item experiment-specific simulations in Tellurium, a Python platform for systems biology \cite{tellurium}
\end{enumerate}

The created \ac{sedml} files were validated in \ac{sedml} WebTools and integrated into the \ac{combine} archive. Subsequently, output files were created by

\begin{enumerate}
    \item simulating all \ac{sedml} files within the \ac{combine} archive using Tellurium and
    \item loading individual \ac{sedml} files in \ac{copasi} or \ac{sedml} WebTools, and included in the \ac{combine} archive.
\end{enumerate}

\subsection*{Assembly of the \acs{combine} archive}

The final assembly of our fully-featured \ac{combine} archive of the Bachmann model was executed in the web platform \textit{CombineArchiveWeb}. We used the same structure as described by Bachmann \textit{et. al.} (see 'setup of an agile working environment'), and extended the scaffold by additional sub-folders containing supplementary files. Thus we created a new empty archive with our credentials on the CombineArchiveWeb with the name \textsf{::bachmann}. The metadata section of our \ac{combine} archive includes a brief description of the study and the contact data of the members.

The folder documentation includes the initial paper from Bachmann \textit{et. al.}  with its additional materials, the paper from Scharm \& Waltemath, and a \textsf{bib.file} with the used references in our project. In the model's folder are the file with the last version of the Bachmann model from 2019 (see Model selection) and the subfolder \textsf{sbgn}. This contains pictures, \ac{sbgnml} files, and other supplemantary files to visualize the model. The folder experiment has Python-scripts, \ac{sedml} files, and other additional files to support the simulation. The results of the different simulations and validations are in the folder result. This folder contains the subfolders Default, Fig(3 - 5), SuppFig9, and SuppFig9\_COPASI; these incorporate \ac{csv} files with tables, PDF files with reports, and pictures of the performed simulations with different tools; the subfolder Validation, in the folder result has the folders Test\_COPASI\_2021-06-28 and Test\_SWT\_2021-06-28, with subfolders and data files of the performed validations from \ac{copasi} and \ac{swt}.

We could upload all data from our repository but report XML files for validations. On this step, the CobineArchiveWeb interface launches the following error: \textbf{Unknown Error: Cannot upload file}, therefore we created a new issue in the GitHub repository from CobineArchiveWeb but did not encounter the settlement of the problem during the project.