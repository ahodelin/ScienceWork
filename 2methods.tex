\section*{Materials and methods} \label{sec:matmet}

The tasks of this project were distributed on  four teams and split into the sections 
\begin{itemize}
    \item Setup of an agile working environment
    \item Systematic review of existing materials and comparison of provided models
    \item Graphical representation
    \item Supply of one or several model scripts
    \item Assembly of the \acs{combine} archive
\end{itemize}

\subsection*{Setup of an agile working environment} \label{setupagwork}
To provide the model from Bachmann \textit{et al.} \cite{bachmannmodel} as a fully featured \ac{combine} archive, we created a \hyperref[https://github.com/ahodelin/Bachmann_Archive]{public repository} using the open-source platform GitHub, with a \ac{cc}0-1.0-license. We chose GitHub as a data management platform to supervise the course of the project as it provides an intuitive and easily customizable environment, along with features for documentation, and agile project management \cite{github}. This repository contains the proposal directory's structure from Scharm \& Waltemath \cite{combine} with the following directories:
\begin{itemize}
    \item Documentation (files describing the model and its characteristics)
    \item Model (files describing and encoding the biological system, e.g. \ac{sbml} format)
    \item Experiment (files describing and encoding the experimental setup, e.g. \ac{sedml})
    \item Result (files obtained from running \textit{in silico} experiments, including plots and tables)
\end{itemize}

\subsection*{Systematic review of existing materials and comparison of provided models}
In our systematic search, we found five \ac{sbml}-Bachmann models in two different repositories, \hyperlink{https://www.systemsmedicine.net/posts/jws-online-biological-systems-modelling}{JWS Online} and \hyperlink{https://www.ebi.ac.uk/biomodels/}{BioModels} (see Baseline data). We chose the latest model dating from 14th November 2019 as it provided complementary files for the simulation.

\subsection*{Graphical representation}

One of the objectives of our project was to provide a standardized graphical representation of the Bachmann model based on the \ac{sbgn}. We performed research one the \hyperlink{https://sbgn.github.io/software}{SBGN website} providing links to archives and databases, but could not find any existing \ac{sbgn} of this model. We decided to create an \ac{sbgn} network \textit{de novo} based on previous works of Le Novère \cite{sbgnnovere} and Touré \textit{et al.} \cite{sbgntoure}. In this step, we selected the \ac{sbgn} language, and lastly created the \ac{pd} map with the web tool Newt Editor (v3.0.3) \cite{newteditor}.

To validate the \ac{sbgnml} we, imported it into several software and libraries including LibSBGN from Newt Editor, the open-source software \ac{vanted} \cite{vanted}, Krayon for \ac{sbgn} \cite{krayon}, and SBGNViz \cite{sbgnviz}. Lastly, we cleaned up the map and colored the relevant features in the model to improve the developed map.

\subsection*{Supply of one or several simulation scripts}

The simulation descriptions to reproduce selected experiments from Bachmann \textit{et al.} \cite{bachmannmodel} in combination with the selected model were provided as \hyperlink{https://sed-ml.org/}{\acs{sedml} Level 1 Version 3 files}. \ac{sedml} files were chosen because they fulfill
the recently published Minimum Information about a Simulation Experiment \ac{MIASE} guidelines (computer-readable exchange format, provision of XML schema). 

An overview of existing tools to generate \ac{sedml} files was obtained from the \hyperlink{http://sed-ml.org/}{\acs{sedml} website}, including a brief description and information on supported model languages. Based on this overview as well as previous experience, we decided to test several different tools in order to create, edit and export \ac{sedml} files for specific experiments using the selected SBML model. 
Initially, we followed the steps taken by Scharm \& Waltemath \cite{combine} to generate a default simulation using \hyperlink{http://sysbioapps.spdns.org/SED-ML_Web_Tools}{\acs{sedml} WebTools Version 2.4} and modified this for a specific experiment in \hyperlink{http://copasi.org/}{\ac{copasi}} v.4.33.246 \cite{copasi}. However, given the type of plots to be created, we eventually decided to generate experiment-speficifc \ac{sedml} files with Tellurium v2.2.0 \cite{tellurium} in order to reproduce the experiments presented by Bachmann \textit{et al.} \cite{bachmannmodel}. We used \hyperref[https://colab.research.google.com/notebooks/welcome.ipynb?hl=de]{Google Colaboratory} (Colab), a cloud-based Jupyter notebook service for collaborative Python programming and loaded our scripts to our GitHub repository. Next, the created \ac{sedml} files were validated using \ac{sedml} WebTools. Finally, all ac{sedml} files (including the associated Colab notebooks, output files (plots and tables) and validation results) were integrated into the \ac{combine} archive.

\subsection*{Assembly of the \acs{combine} archive}

The development of the \ac{combine} Archive was performed in accordance with a recently published guideline from Schwarm \& Waltemath \cite{combine}.

The final assembly of our fully-featured \ac{combine} archive of the Bachmann model was executed in the platform \ac{webcat}. We used the same structure as described by Schwarm \& Waltemath (see 'setup of an agile working environment'), and extended the scaffold by additional sub-folders containing supplementary files. Thus we created a new empty archive with our credentials on the \ac{webcat} with the name \textsf{::bachmann}. The metadata section of our \ac{combine} archive includes a brief description of the study and the contact data of the members.

The folder 'documentation' includes the initial paper from Bachmann \textit{et al.}  with its additional materials, the paper from Scharm \& Waltemath, and the file 'SystemsBiology.bib' with the used literature in our project. 'model' contains files with the latest version of the Bachmann model from 2019 (see 'Model selection') and the sub-folder 'sbgn', it contains pictures, \ac{sbgnml} files, and other supplementary files to visualize the model. The folder 'experiment' provides Python-scripts, \ac{sedml} files, and other additional files to support the simulation. The results of the different simulations and validations are in the folder 'result'. This folder also contains the sub-folders 'Default', 'Fig(3-5)', 'SuppFig9', and 'SuppFig9\_COPASI' that incorporate \ac{csv} files with tables, PDF files with reports, and pictures of the performed simulations with different tools.  'result' contains a sub-folder 'validation'  including the additional folders 'Test\_ COPASI\_2021-06-28' and 'Test\_SWT\_2021-06-28' with data files of the performed validations from \ac{copasi} and \ac{swt}.

The Archive can be accesses via the link in the 'appendices'-section.